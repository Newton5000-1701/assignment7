
\documentclass[letter,12pt]{article}
\setcounter{secnumdepth}{0}
\usepackage{pawlowski}
\usepackage{natbib}
\usepackage[margin=1.0in]{geometry}                         
\usepackage{graphicx}
\usepackage{amssymb}
\usepackage{listing}
\usepackage{amsmath}
\usepackage{verbatim}
\usepackage{epstopdf}
\usepackage{epstopdf}
\begin{document}



\begin{center}
{\Large \bf PHYS 380 - Fall 2024}\\
\vspace{10pt}
{\bf Assignment 7}\\
\vspace{10pt}
{Isaac Thompson}\\
\vspace{10pt}
{$11$ November $2024$}
\end{center}
\vspace{0.1in}

\section{Problem 1}
\vspace{0.15in}

\indent In order to derive approximation methods for derivatives, one must recall the Taylor series in the following form: for $x\in\mathbb{R}$, and any $C^{\infty}$ function, $f:\mathbb{R} \to \mathbb{R}$, with finite quantity $h \in \mathbb{R}$, the function $f$ translated by an amount $h$ can be written precisely as
\begin{align}
f(x+h) = \sum_{n=0}^{\infty} \frac{1}{n!} \frac{d^n f(x)}{dx^n} h^n
\end{align}

\noindent We also note that $f(x-h)$ is equivalent to the above with the stipulation that $h \to -h$, since $h$ is an arbitrary finite real number. For an approximation to second order in $h$, consider the following:

\begin{align}
f(x+h) - f(x-h) = \left( f(x) + \frac{df}{dx}h + \frac{d^2 f}{dx^2}h^2 \right) - \left( f(x) - \frac{df}{dx}h + \frac{d^2 f}{dx^2} h^2 \right),
\end{align}
where we have neglected terms of order $h^3$ and higher. This equation can be see to be equivalent to 
\begin{align}
\frac{df}{dx} = \frac{f(x+h) - f(x-h)}{2h},
\end{align}
and the equality holds \emph{only} in the second order approximation. For sufficiently small $h$, we conclude that the derivative of $f$ is approximately,
\begin{align}
\frac{df}{dx} \approx \frac{f(x+h)-f(x-h)}{2h}
\end{align}
\noindent In order to derive a third order approximation, we consider that $f(x+\Delta x)$, where $\Delta x = \alpha h$ for integer $\alpha$, is, to third order in $h$,
\begin{align}
f(x+\Delta x) = f(x) + f'(x)\Delta x + \frac{1}{2} f''(x) \Delta x ^2 + \frac{1}{6} f'''(x) \Delta x ^3,
\end{align}
where ``$\ ' \ $" is understood to denote a derivative with respect to $x$. It can be shown that the following combination of $f(x+h)$, $f(x-h)$, $f(x+2h)$, and $f(x-2h)$, where appropriate values of $\alpha$ are picked by comparison with equation $5$,  produces an approximation of $df/dx$ accurate to third order in $h$:
\begin{align}
f'(x) = \frac{8 ( f(x+h)-f(x-h) ) - ( f(x+2h)-f(x-2h) ) }{12h}.
\end{align}
Again, we note that the equality holds \emph{only} in the third order approximation, otherwise, we conclude that for small $h$, $df/dx$ may be approximated as 
\begin{align}
f'(x) \approx \frac{8 ( f(x+h)-f(x-h) ) - ( f(x+2h)-f(x-2h) ) }{12h}.
\end{align}

\section{Problem 2}
\vspace{0.15in}

This problem asks for a generalization of the software written in a previous problem which calculates the scalar potential and the electric field of a single point charge. The problem essentially consists of recalling that the wave equation, $\Box \phi = J^0$, is linear, and so the potential of a distribution of point charges is a linear combination of the contribution of each charge:
\begin{align}
\phi = \sum_{i=1}^N \phi_i = \sum_{i=1}^N \frac{q_i}{4\pi \epsilon_0 r_i},
\end{align}
where $N$ is the number of point charges, and $\phi_i$ is the Coulomb potential of the $i$th charge. So, in principle, if the charges and their positions are known, the potential of the system is known. The electric field is given by $E^j = -\partial_i \phi \delta^{ij}$:
\begin{align}
\mathbf{E} = -\nabla \phi
\end{align}
Employing the provided random point charge distribution software to modify the electric potential software previously written, figures $1$ and $2$ were obtained. The results are consistent with expectations; the expectations being a random field of sources and sinks of various strengths. Due to the in-built randomness, there is no way of predicting a specific outcome, other than that one will see a distribution of sources and sinks. 

\begin{figure}[htbp]
    \centering
    \includegraphics[width=0.7\textwidth]{potential.png}
    \caption{The scalar potential of a random distribution of $50$ point charges with $|q|<1$.}
\end{figure}

\begin{figure}[htbp]
    \centering
    \includegraphics[width=0.7\textwidth]{efield.png}
    \caption{The electric field of a random distribution of $50$ point charges with $|q|<1$.}
\end{figure}

\section{Problem 3}

This problem asks for a solution to the damped harmonic oscillator via Euler's method. The statement of the problem is as follows:
\begin{align}
\frac{d^2 y}{dt^2} + \gamma \frac{dy}{dt} + \omega^2 y = 0,
\end{align}
where $\gamma$ is the damping parameter, and $\omega$ is the natural frequency of the system. We must be careful with the different cases. For the critically damped case, the equation of motion becomes,
\begin{align}
\frac{d^2 y}{dt^2} + 2\omega \frac{dy}{dy} + \omega^2 y = 0,
\end{align}
which has the general solution:
\begin{align}
y(t) = c_1 e^{-\omega t} + c_2 t e^{-\omega t}.
\end{align}
In the case of no damping, the equations of motion reduce to the well known simple harmonic oscillator, with general solution:
\begin{align}
y(t) = y(0) \cos(\omega t).
\end{align}
In the cases of under and over damping, the characteristic equations of the system are
\begin{align}
k_{ud,\pm} = \frac{-\gamma \pm i\sqrt{4\omega^2 - \gamma^2}}{2},
\end{align}
and
\begin{align}
 k_{od,\pm} = \frac{-\gamma \pm \sqrt{\gamma^2-4\omega^2}}{2}.
\end{align}
(Note: these two $k$'s are mathematically the same; writing them out this way forces the radicand to be positive definite.) Now, if the initial conditions are $y(0)=2$ and $\dot{y}(0)=0$, then the exact solutions are
\begin{enumerate}
\item{$\gamma = 0$: $y(t)=2\cos(\omega t)$}
\item{$\gamma = 2\omega$: $y(t) = 2 e^{-\omega t} (\omega t + 1)$}
\item{$\gamma < 2\omega = 2$: $y(t) = 2 e^{-t} \left(\cos(\frac{1}{2} t\sqrt{4\omega^2 - \gamma^2})+\frac{\gamma}{\sqrt{4\omega^2 - \gamma^2}}\sin(\frac{1}{2} t\sqrt{4\omega^2 - \gamma^2})\right)$
\item{$\gamma > 2\omega = 3$: $y(t) = 2 e^{-\frac{3t}{2}} \left(\cosh(\frac{1}{2} t\sqrt{\gamma^2-4\omega^2})+\frac{\gamma}{\sqrt{\gamma^2-4\omega^2}}\sinh(\frac{1}{2} t\sqrt{\gamma^2-4\omega^2})\right)$
\end{enumerate}

The first case is the simple harmonic oscillator which is a cosine wave of amplitude $2$. Note, in figure $3$, that the Euler method is insufficient as it predicts an increasing system energy, which is unphysical; the Runge-Kutta method is required to ensure a physical result. The rest of the cases are the interesting ones, and they all match the expectations of the analytical solutions. The critically damped solution represents the situation in which the frequency associated with damping, $\omega_d := \gamma/2$ is precisely the natural frequency $\omega$. The under and over damped scenarios occur due to the relation between $\omega$ and $\omega_d$: if the ``damping frequency" is greater than the natural frequency, the system will rapidly approach zero without oscillating, whereas, if the ``damping frequency" is less than the natural frequency, the system will oscillate around zero with a decreasing, time dependent amplitude. In our underdamped case, it can be shown that the amplitude is 
\begin{align}
A(t) = 2\omega e^{-\gamma t /2}
\end{align}
Thus, the amplitude decays at a rate 
\begin{align}
|\dot{A}(t)| = \gamma \omega e^{-\gamma t /2}
\end{align}
With this, and recalling that the energy of a harmonic oscillator is 
\begin{align}
E = \frac{1}{2} m \omega^2 A,
\end{align}
we can determine the rate at which the oscillator loses energy, as the result is directly proportional to $\dot{A}(t)$:
\begin{align}
|\dot{E}| = \frac{1}{2} m \gamma \omega^3 e^{-\gamma t / 2} 
\end{align}
This confirms the expectation that the system will lose energy exponentially. 


\begin{figure}[htbp]
    \centering
    \includegraphics[width=0.7\textwidth]{no_damping_comparison.png}
    \caption{Numerical and analytical solution to the undamped harmonic oscillator.}
\end{figure}

\begin{figure}[htbp]
    \centering
    \includegraphics[width=0.7\textwidth]{critical_damping_comparison.png}
    \caption{Numerical and analytical solution to the critically damped harmonic oscillator.}
\end{figure}

\begin{figure}[htbp]
    \centering
    \includegraphics[width=0.7\textwidth]{underdamping_comparison.png}
    \caption{Numerical and analytical solution to the underdamped harmonic oscillator.}
\end{figure}

\begin{figure}[htbp]
    \centering
    \includegraphics[width=0.7\textwidth]{overdamping_comparison.png}
    \caption{Numerical and analytical solution to the overdamped harmonic oscillator.}
\end{figure}

\section{Problem 4}

This problem asks for a solution to the differential equation of motion describing a bicycle traversing a flat surface. We consider the system's energy. The bicycle is being provided energy by the human riding it at a rate, $P$, which we assume to be constant. The energy of the system is entirely kinetic (flat surface), and so we write:
\begin{align}
E = \frac{1}{2} m v^2 \to \dot{E} = mv\dot{v}
\end{align}
The equation of motion is then, recalling that $\dot{E} = P$,
\begin{align}
\dot{v} = \frac{P}{mv}
\end{align}
The solution is plotted in figure $7$. We see that the velocity resembles a square root, which implies that the velocity will increase indefinitely (assuming the human can) but the acceleration of the bicycle asymptotically approaches zero. This makes sense, as we are, via the human, giving the system energy at a constant rate; but for any given mass, more energy is required to accelerate it further as velocity increases. 

\begin{figure}[htbp]
    \centering
    \includegraphics[width=0.7\textwidth]{bicycle.png}
    \caption{Numerical solution to the bicycle equation of motion.}
\end{figure}

\end{document}




